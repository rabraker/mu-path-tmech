\documentclass[a4paper,twoside]{article}
\usepackage[margin=1.5in]{geometry}
\begin{document}
\hspace*{0.45\linewidth}
\begin{minipage}{0.52\linewidth}
Roger A. Braker\par
University of Colorado, Boulder\par
Dept of Electrical, Computer and Energy Engineering \par
425 UCB\par
Boulder, CO 80305\par
United States\par
roger.braker@colorado.edu\par
\today
\end{minipage}
\par\bigskip

Dear Prof. Qingze Zou,\par\bigskip
% check this is correct

It is our pleasure to submit this manuscript entitled \textit{Improving the Image Acquisition Rate of an Atomic Force Microscope Through Sub-sampling and Reconstruction} (authored by Roger A. Braker, Yufan Luo, Lucy Y. Pao and Sean Andersson) for consideration to be published in the Transactions on Mechatronics. This paper is submitted for possible publication in the Focused Section on Nano/Micro Motion  Systems:  Design,  Sensing  and  Control(NMMS).

The main goal of this manuscript is to show that the native imaging speed of an atomic force microscope (AFM) can be improved by under-sampling  the specimen (i.e., only acquiring a subset of the total pixels for a target image size). While sub-sampling schemes have seen increased attention recently, many of the existing studies are based purely on simulation. However, the AFM is fundamentally a mechanical device and most of these sub-sampling studies provide little insight
into how the instrument will respond to the proposed sampling patterns nor into how the patterns might be implemented. Our manuscript seeks to partially fill this gap, as it largely focuses on the practical challenges which must be overcome to realize a sub-sampling scheme.

The manuscript has not been published nor is it under review elsewhere. All four authors agree to submit the manuscript in its present form to the IEEE/ASME Transactions Mechatronics focused section NMMS. This work was supported in part by Agilent Technologies, Inc, The US National Science Foundation (under Grants CMMI-1234980, CMMI-1234845, CMMI-1562031, and DBI-1352729), and the Hanse Wissenschaftskolleg in Delmenhorst, Germany. Thank you for the consideration and we look forward to hearing from you.

\par\bigskip
\noindent Sincerely,
\par\bigskip
\noindent R.A. Braker\par
\noindent Y. Luo\par
\noindent L.Y. Pao\par
\noindent S. Andersson

\end{document}
%%% Local Variables:
%%% mode: latex
%%% TeX-master: t
%%% End:

\documentclass[a4paper,twoside]{article}
\usepackage[margin=1.5in]{geometry}
\begin{document}
\hspace*{0.45\linewidth}
\begin{minipage}{0.52\linewidth}
Roger A. Braker\par
University of Colorado, Boulder\par
Dept of Electrical, Computer and Energy Engineering \par
425 UCB\par
Boulder, CO 80305\par
United States\par
roger.braker@colorado.edu\par
\today
\end{minipage}
\par\bigskip

Dear Prof. Qingze Zou,\par\bigskip
% check this is correct

It is our pleasure to submit this manuscript entitled \textit{Improving the Image Acquisition Rate of an Atomic Force Microscope Through Sub-sampling and Reconstruction} (authored by Roger A. Braker, Yufan Luo, Lucy Y. Pao and Sean Andersson) for consideration to be published in the Transactions on Mechatronics. This paper is submitted for possible publication in the Focused Section on Nano/Micro Motion  Systems:  Design,  Sensing  and  Control (NMMS).

The main goal of this manuscript is to show that the native imaging speed of an atomic force microscope (AFM) can be improved by under-sampling the specimen (i.e., only acquiring a subset of the total pixels for a target image size). While sub-sampling schemes have seen increased attention recently, many of the existing studies are based purely on simulation. However, the AFM is fundamentally a mechanical device and most of these sub-sampling studies provide little insight
into how the instrument will respond to the proposed sampling patterns nor into how the patterns might be implemented. Our manuscript seeks to partially fill this gap, as it largely focuses on the practical (yet non-trivial) challenges which must be overcome to realize a sub-sampling scheme. 

The manuscript is an extension of two prior conference papers (ACC 2014 and ACC 2018). That first paper (ACC 2014) introduced a sub-sampling technique we called $\mu$-path scanning, which consists of short, randomly placed scans. However, we only evaluated its effectiveness via simulation. In the second conference paper (ACC 2018), we described our initial efforts at experimental implementation of $\mu$-path scanning, which used only primitive integral control. Here, we design and implement higher bandwidth feedback controllers. Further, we identify coupling between the horizontal plane and the $Z$-axis as one of the primary limiting factors in $\mu$-path scanning, which we mitigate with a feedforward control scheme. A second significant difference between this manuscript and our prior work is that we compare $\mu$-path scanning to a much simpler sub-sampling technique, where we scan every fourth or every eighth line, followed by an interpolation. Finally, we also introduce a new reconstruction method designed to mitigate artifacts arising from the $\mu$-path pattern.


The manuscript has not been published nor is it under review elsewhere. All four authors agree to submit the manuscript in its present form to the IEEE/ASME Transactions on Mechatronics focused section on NMMS. This work was supported in part by Agilent Technologies, Inc., the US National Science Foundation (under Grants CMMI-1234980, CMMI-1234845, CMMI-1562031, and DBI-1352729),
a Palmer Endowed Chair Professorship,
and the Hanse Wissenschaftskolleg in Delmenhorst, Germany. Thank you for the consideration and we look forward to hearing from you.

\par\bigskip
\noindent Sincerely,
\par\bigskip
\noindent R.A. Braker\par
\noindent Y. Luo\par
\noindent L.Y. Pao\par
\noindent S. Andersson

\end{document}
%%% Local Variables:
%%% mode: latex
%%% TeX-master: t
%%% End:

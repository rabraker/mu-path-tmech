




\begin{enumerate}
\item \textbf{Initialization:} In this state, the user uses the PicoView
  software to bring the tip to the sample with the stepper motor. Once the tip
  is engaged in the surface, the user switches the control from PicoView to the
  custom software in LabView. The system writes parameters including the
  locations of $\mu$-paths from a desktop file to a host-to-FPGA FIFO (first in,
  first out) buffer. Next, the closed-loop control of all the axes through the
  FPGA is enabled and the system performs the following operations.
    \todo{Shorten this. Sean, what was your trick to not indent enumerate?}
  \begin{itemize}
  \item \textbf{Read} the starting location of the first $\mu$-path from the
    host-to-FPGA FIFO.
  \item \textbf{Set} the $z$-axis loop to position control mode.
  \item \textbf{Wait until} $|e_z(k)|$ reaches a settling criterion, then go to
    state 2.
  \end{itemize}
\item \textbf{$xy$-axis move:} The system moves the tip to the next target
  location by holding $r_X$ and $r_Y$ to the beginning of the next $\mu$-path.
\item \textbf{Tip engage:} We set the $Z$ setpoint to $r_{Z,s}$ so that the
  tip is driven towards the sample surface. 
  \begin{itemize}
  \item \textbf{Set} the $z$-axis loop to deflection control mode.
  \item \textbf{Wait until} $|e_z(k)|$ reaches a settling criterion, then
    transition to state 4. \todo{Update this}
  \end{itemize}
\item \textbf{Scan:} In this state, the system moves the tip following the
  trajectory of the current $\mu$-path.
  \begin{itemize}
  \item \textbf{Update} $x_{ref}$, and $y_{ref}$ iteratively to follow the
    trajectory of the current $\mu$-path to the end. At the same time, write the
    $x,y,z,$ and deflection measurements to an FPGA-to-host FIFO.
  \item \textbf{Wait until} $|e_x(k)|$ and $|e_y(k)|$ reach the end of the
    $\mu$-path,
    % a settling criterion,
    then go to state 5.
  \end{itemize}
  % \item \textbf{Transition decision:} The tip is at the end of the $\mu$-path
  %   at this time. The location of the next $\mu$-path is read from the FIFO.
  %   In
  %   order to accelerate the scanning process, the decision to lift the tip or
  %   not is made in this state. The time $t_{xy\_scan}$ it takes to scan to the
  %   beginning of the next $\mu$-path is compared to
  %   $t_{\textit{z}up}+t_{xy}+t_{\textit{z}down}$. If the former is larger,
  %   then
  %   go to state 6; otherwise, go to state 2.
\item \textbf{Tip withdraw:} In this state, the system withdraws the tip in the
  $z$-direction. A large $K_I$ is used for the $z$-piezo here in order to save
  time.
  \begin{itemize}
  \item \textbf{Set} the $z$-axis loop to position control mode.
  \item \textbf{Set} the $z$ setpoint at a specified distance off the surface.
  \item \textbf{Wait until} $|e_z(k)|$ reaches a settling criterion, then go to
    state 2.
  \end{itemize}
\end{enumerate}

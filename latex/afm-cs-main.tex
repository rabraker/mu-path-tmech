\documentclass[twocolumn,twoside]{IEEEtran/IEEEtran}
% \documentclass[defaultstyle,11pt]{IEEETran}
% Lucy says to look at:
% float_page_fraction
% tex_page_fraction
% https://tex.stackexchange.com/questions/35125/how-to-use-the-placement-options-t-h-with-figures#35130
% \renewcommand\floatpagefraction{0.5} %% default: 0.5
% \renewcommand\topfraction{0.7} % default: 0.7
% \renewcommand*\textfraction{.05}
\usepackage{amssymb}		% to get all AMS symbols
\usepackage{amsthm}
\usepackage{mathtools}
\usepackage{graphicx}		% to insert figures
\usepackage{hyperref}		% PDF hyperreferences??
\usepackage{multirow}
\usepackage{mathtools}
\usepackage{cite}
\usepackage{listings}
\usepackage{booktabs}
% \usepackage{amsthm}
\usepackage[percent]{overpic}
\usepackage{caption}
\usepackage{subcaption}
\usepackage{xspace}
\usepackage{enumitem}
\usepackage{catchfilebetweentags} % for data input
\usepackage[inkscapelatex=false, inkscapepath=svgsubpath]{svg}
\usepackage{url}
\usepackage{mdframed}
\usepackage{amsmath, dsfont, bbold}
% \usepackage{algpseudocode,algorithm,algorithmicx}
\usepackage[linesnumberedhidden,ruled]{algorithm2e}
\usepackage{placeins}
\usepackage{todonotes}

% Indent paragraphs inside enumerate
\usepackage{enumitem}
\setlist{  
  listparindent=\parindent,
  parsep=0pt,
}
\newcommand{\xyc}{\ensuremath{XY}\xspace}
\newcommand{\xc}{\ensuremath{X}\xspace}
\newcommand{\yc}{\ensuremath{Y}\xspace}
\newcommand{\zc}{\ensuremath{Z}\xspace}


\usepackage{layouts}
%textwidth: \printinunitsof{in}\prntlen{\textwidth}
%linewidth: \printinunitsof{in}\prntlen{\linewidth}
% textwidth: \printinunitsof{in}\prntlen{\textheight}

% this reveals that textwidth and linewidth are 6.5 inches,
% and textheight is 9.1 inches.





%%%%%%%%%%%%%%%%%%%%%%%%%%%%%%%%%%%%%%%%%%%%%%%%%%%%%%%%%%%%%%%%%
%%%%%%%%%%%%%%%       BEGIN DOCUMENT...         %%%%%%%%%%%%%%%%%
%%%%%%%%%%%%%%%%%%%%%%%%%%%%%%%%%%%%%%%%%%%%%%%%%%%%%%%%%%%%%%%%%

\begin{document}
\title{High-speed atomic force microscopy through $\mu$-path sampling}

\author{Roger A. Braker and Yufan Luo and Lucy Y. Pao and Sean B. Andersson
  \thanks{The authors are with the Dept. of Electrical, Computer, and Energy
    Engineering at the University of Colorado, 425 UCB, Boulder, CO 80309,
    United States. Phone: +1 (303) 492-2360. Fax: +1 (303) 492-2758. R. A.
    Braker (corresponding author roger.braker@colorado.edu) is a graduate
    student and L.Y. Pao (pao@colorado.edu) is the Richard \& Joy Dorf
    Professor.} \thanks{This work was supported in part by the US National
    Science Foundation (NSF Grant CMMI-1234980) and Agilent Technologies, Inc.}
}

\maketitle
\begin{abstract}
  Undersampling-based approaches in Atomic Force Microscopy (AFM) aim to reduce
  the time to acquire an image by reducing the number of measurements needed
  while still maintaining image quality. In this paper, we describe a hardware
  implementation and demonstration of one such approach based on the use of
  collections of short scans known as $\mu$--paths. Using a commercial AFM, we
  acquire data on two different grating samples using two different sampling
  approaches: randomly-placed $\mu-$paths and $\mu-$paths designed to minimize
  the reconstruction error based on partial prior information about the spatial
  frequency content in the sample image. Reconstructions are made from these
  data using three approaches: (1) inpainting, an interpolation technique that
  diffuses information from sampled locations into unsampled regions, (2)
  basis-pursuit, a compressive sensing-based algorithm that seeks to minimize a
  measure of sparsity in the underlying image, and (3) a new variant of basis
  pursuit, Basis Pursuit Vertical Variation (BPVV), that is designed to reduce
  artifacts arising from the sampling pattern. The quality of the resulting
  images is compared to images from standard raster scans of the same regions at
  comparable imaging rates using both the peak signal-to-noise ratio and the
  structured similarity index metric. These experiments demonstrate that at slow
  scan rates, the $\mu-$path scheme produces similar image quality with
  significantly less sampling (and thus less tip-sample interaction) and that at
  high rates, the undersampling scheme produces higher quality images than
  raster scanning.
\end{abstract}

\section{Introduction}\label{sec:introduction}
The Atomic Force Microscope (AFM) is a powerful instrument capable of imaging
sample surface topography, material characteristics, and other surface
properties, at the nanometer scale. AFMs acquire information about the sample
through a variety of imaging modes, all of which rely on the deflection of the
cantilever caused by the tip-sample interaction force. While the details of each
of the modes are quite different, in general, feedback control is used to hold
the deflection signal constant and information about the property of interest is
inferred from the applied control or the measured dynamics of the cantilever
\cite{Abramovitch:2007gt}. Because of its versatility, spatial resolution, and
ability to image in vacuum, air, and liquid, AFM is widely used in a variety of
disciplines, including physics, biology and materials science
\cite{Dufrene:2017gm,Yang:2017im,Payton:2016hj,Altman:2015ic,Haase:2015eh}
	
In addition to imaging static samples, AFM is increasingly being applied to
study dynamics in systems with nanometer-scale features
\cite{Yang:2017im,Shibata:2017da,Shibata:2015jd,Ando:2014ja}. The image
acquisition time of conventional AFMs, however, is typically on the order of
seconds to minutes, severely limiting the time-scales that can be explored.
Driven by the need for faster imaging, there continue to be many active research
efforts to overcome this challenge. Since an AFM is fundamentally a mechanical
microscope, many of the approaches to high-speed AFM (HS-AFM) have focused on
modifications to the physical components. These have included the use of small,
fast cantilevers \cite{viani1999fast, braunsmann2010high,Adams:2016hg}, the
development of faster actuators \cite{Maroufi:2015gt,Yong:2012kd,Kenton:2012cm},
as well as the application of advanced controllers
\cite{Rana:2018es,Yong:2015gr,butterworth2010adaptive,salapaka2002high}. By
combining different techniques, there are current generation, high-end
instruments that can image at rates on the order of 1-10 frames/sec
\cite{Ando:2014ja}. However, the fastest rates are achieved only over small scan
ranges and there remain many systems of interest whose dynamics are faster even
than these instruments can achieve. In addition, there is a large installed base
of much slower AFMs that can benefit from alternative approaches to improving
imaging rate.
	
A complimentary class of HS-AFM approaches considers modifications to the
sampling scheme rather than the system mechanics. Alternative scan trajectories
like spirals, cycloids, and Lissajous figures, have been used in place of the
standard raster scan. These trajectories are easier for the actuators to follow,
allowing the tip to be moved more quickly and leading to faster image
acquisition on an otherwise unmodified instrument
\cite{Bazaei:2017dm,Wu:2015dt,Rana:2014bj,Tuma:2012hv,Yong:2010gm}. Such
approaches are ultimately limited by the vertical bandwidth of the instrument
for regulating the deflection signal \cite{Teo:2016ev}. Related to these are
local scanning methods that use the measurements in real-time to steer the tip
to focus the scan on features of interest, reducing imaging time by reducing the
amount of sampling needed
\cite{Hartman:2018ja,Wen:2018fl,Zhang:2015cf,Huang:2014dw}. While these have
been shown to yield an order of magnitude or better improvement in imaging rate,
they are limited in the class of samples that can be imaged.
	
An alternative group of non-raster scanning schemes, introduced in
\cite{song2011video,andersson2012non}, is based on the idea of undersampling. As
with local scanning, data acquisition time is reduced by reducing the amount of
measurements acquired. Unlike local scanning methods, this approach still
presents the full image. Taking advantage of the redundancy in many natural
signals of interest, the final surface image can be recovered from the limited
number of measured pixels using a variety of different reconstruction methods
such as inpainting or schemes based on the theory of compressive sensing (CS)
\cite{chen2013enhancement,luo2015comparison}. In addition to the reduced imaging
time, undersampling schemes also reduce tip-sample interactions, thereby
reducing the likelihood of tip wear or sample damage.
	
One simple way to create undersampling schemes is by modifying existing full
scan patterns including raster, spiral, and Lissajous scanning. For example,
subline sampling is generated by randomly skipping some of the horizontal lines
in raster scanning \cite{han2018reconstruction,chen2013enhancement}. For spiral
and Lissajous patterns, the scan parameters (frequency and amplitude) can be
selected to ensure the trajectory only passes through a desired fraction of the
pixels in the final image. The scanning time for these smooth undersampling
patterns can be estimated based on the proportion of the pixels in the
trajectory. However, results in the literature of CS make clear that
\textit{randomness} in the sampling pattern is essential for creating good
reconstructions in the general case. The smooth nature of these patterns reduces
their randomness and thus leads to less accurate reconstructions than a
completely random sampling of the pixels for the same sampling fraction.
	
In AFM, implementing a truly random pattern requires that the tip be engaged
with the surface to collect a measurement, lifted and moved to the new location,
and engaged again. Because the re-engagement process is typically slow, this can
lead to excessively long image acquisition times and negate the gains from
undersampling \cite{andersson2012non}. One discrete undersampling scheme, called
a $\mu$-path pattern, was proposed in \cite{maxwell2014compressed} to improve
sampling efficiency. The $\mu$-path pattern consists of randomly placed short,
horizontal scans (see Fig.~\ref{fig:mu_mask} for one such pattern) and is
designed to balance the randomness needed to ensure good reconstruction with
continuous scanning to reduce the number of tip
engagements. %Though the pattern is motivated from CS, it also works well in practice for inpainting reconstruction \cite{luo2015comparison}.
	
Our previous work on the $\mu$-path pattern using theoretical calculations,
simulations, and a preliminary implementation, demonstrates that the approach
can achieve significant scanning time reduction while maintaining faithful image
reconstruction \cite{maxwell2014compressed,Luo:2015tu,Braker:2018gi}. This paper
builds upon that earlier work in several significant ways. First, we demonstrate
performance using both randomly-placed $\mu-$paths and designed $\mu-$paths that
are optimized to minimize the expected reconstruction error given some initial
estimate of the frequency structure of the sample. These schemes are described
in Sec. \ref{sec:samplingScheme}. We also introduce a new reconstruction
algorithm which is designed to minimize the artifacts arising from the structure
of the $\mu$-path scanning pattern. This algorithm is described and compared to
existing reconstruction methods through simulations in Sec.
\ref{sec:reconstructionMethods}. We then demonstrate the entire scheme from
sampling through reconstruction using experiments on a commercial AFM (Agilent
5500). The experimental setup and implementation details are described in Secs.
\ref{sec:experimentalSetup} and \ref{sec:implementation}\footnote{The software
  portion of the implementation and image reconstruction described in this paper
  can be found at \url{https://github.com/yufan88/}}. Experimental results with
two different grating samples are provided in Sec.
\ref{sec:experimentalResults}. Finally, we provide a few concluding remarks in
Sec. \ref{sec:conclusions}.


% =========================================================================
\subsection{Reconstruction methods}
\label{sec:reconstructionMethods}
	
In this work, we use three reconstruction algorithms to generate the final
surface image. The first one comes from the field of compressive sensing (CS).
CS is a signal processing technique which aims at signal reconstruction from a
relatively small (sub-Nyquist limit) number of measurements
\cite{carmi2014compressive}. It takes advantage of the approximate sparsity of
real-world signals, that is, of the fact that many coefficients of such signals
are close to zero when represented in an appropriate basis. CS methods seek the
true image signal $x\in\mathbb{R}^n$ from the following observation equation,
\begin{equation}\label{op:observation}
  y = \Phi x = \Phi\Psi\eta,
\end{equation}
\noindent where $y\in\mathbb{R}^m$ is the observation vector, $\Phi$ is an
$m\times n$ matrix defining the measurements, $\Psi$ is an $n\times n$ sparsity
basis and $\eta$ is the sparse representation of $x$ in the domain of $\Psi$. In
general, $m\ll n$. In an imaging application, where the underlying signal is a
matrix $X\in\mathbb{R}^{h\times s}$, we take $x=\text{vec}(X)$, where the
$\text{vec}(\cdot)$ operator stacks the columns of a matrix. Note that the
symbols $x$ and $y$ are also used to indicate coordinate directions in the AFM.
They are reused here as they are standard notation in the CS literature; their
specific meaning throughout the paper should be clear from context.
	
In the AFM application, the probe can only measure a single pixel at a time.
This implies that $\Phi$ is a sparse matrix with each row having only one
nonzero entry and that $y$ is a subset of $x$. Ideally, the sparsity basis and
the measurement matrix $\Phi$ will have a low \textit{mutual coherence}, a
measure that describes how each of the rows of $\Phi$ (the measurements)
``spreads out'' in the domain of $\Psi$ \cite{candes2007sparsity}. In practice,
the Discrete Cosine Transform (DCT) keeps a good balance between achieving a low
mutual coherence between the basis and the required structure of the AFM sensing
matrices and providing a high sparsity of typical AFM sample images. Therefore,
for concreteness, in this work we use the DCT as the sparsity basis $\Psi$.
	
One common realization of the CS-based reconstruction problem, known as basis
pursuit with denoising (BPDN), is given by the following optimization problem,
\begin{equation}\label{op:bp}
  \begin{aligned}
	\min_{x} \left \| \Psi^{-1}x \right \|_1\quad \text{subject to}\quad \|\Phi
    x - y\|_2 < \sigma,
  \end{aligned}
\end{equation}
where $\| \cdot \|_1$ and $\| \cdot \|_2$ denote $l_1$ and $l_2$ norms,
respectively, and $\sigma$ represents some measure of the signal noise such as
its variance. This problem essentially searches for the sparsest signal from all
the candidates that match the measurements. While BPDN is a computationally
demanding algorithm, particularly for large images, it in general provides
accurate reconstructions. In this work, we use the package
$l_1$-magic~\cite{l1magic} to solve the BPDN problem. More efficient greedy
algorithms have also been developed. One of these, Simplified Matching Pursuit
(SMP), was created by two of the authors and was designed specifically for the
sampling matrices in the AFM setting \cite{Luo:2015tu}.
	
Although BPDN is effective in the general setting, using it for reconstruction
from horizontal $\mu$-path samples often yields artifacts in the vertical
direction (that is, the direction orthogonal to the $\mu$-path scans) leading to
strong discontinuities in the image. These artifacts grow more prominent as the
length of the $\mu$-paths is increased \cite{maxwell2014compressed} as these
longer paths lead to increased mutual coherence between the measurement matrix
$\Phi$ and the sparsity basis $\Psi$. However, longer $\mu$-paths means fewer
tip re-engagements and thus shorter scan times. As a result, the second
algorithm we consider is a new variant of BPDN that we term basis pursuit with
vertical variation (BPVV). In order to mitigate the vertical artifacts, BPVV
adds a vertical total variation penalty in the spatial domain to the
optimization objective. That is, \eqref{op:bp} is modified to
\begin{equation}\label{op:bp_tv}
  \begin{aligned}
	\min_{x} \left \| \Psi^{-1}x \right \|_1 + \epsilon \left \| \nabla_v x
    \right \|_1 \, \text{subject to} \, \| \Phi x - y \|_2 < \sigma,
  \end{aligned}
\end{equation}
where ${\nabla_v x=\text{vec}(\nabla_v X)}$ is the discrete gradient in the
vertical ($y$) direction and
% \textcolor{red}{
\begin{equation}
  (\nabla_v X)_{i,j} = 
  \begin{cases}
    X_{i+1,j} - X_{i,j}, & 1\leq i<h,~ 1\leq j\leq s,\\
    0, & \phantom{:}i=h,~\phantom{:::::} 1\leq j\leq s.
  \end{cases}
\end{equation}


\begin{algorithm}
  \SetKwInOut{Initialize}{Initialize} \SetKwInOut{Output}{Output}
		
  \Initialize{$x = d_y = w = b_y = b_w = k = 0$, $y_0=y$ sampled pixels}
  \While{$\left \|\Phi x^k - y^k \right \|_2 \ge \sigma$} {
    % $x^{k+1} = x^{k}$\\
    \For{$i= 1$ \KwTo $N$} {
      $p\leftarrow\mu\Phi^Ty^k+\lambda\nabla_v^T\left(d_y-b_y\right)+\gamma\Psi\left(w-b_w\right)$\\
      $x^{k+1}\leftarrow \left(\mu\Phi^{T}\Phi+\lambda\nabla_v^T\nabla_v+\gamma I\right)^{-1}p$\\
      $d_y\leftarrow \text{shrink}(\nabla_vx^{k+1}+b_y,1/\lambda)$\\
      $w\leftarrow \text{shrink}\left(\Psi^{-1}x^{k+1}+b_w,1/\gamma\right)$\\
      $b_y\leftarrow b_y+\left(\nabla_vx^{k+1}-d_y\right)$\\
      $b_w\leftarrow b_w+\left(\Psi^{-1}x^{k+1}-w\right)$\\
    }
    $y^{k+1} = y^k + y - \Phi x^{k+1}$\\
    $k \leftarrow k + 1$\\
  } \Output{Reconstruction $x^k$}
  \caption{Split Bregman BPVV}
  \label{ag:BPTV}
\end{algorithm}
\todo{Modify this for NESTA}

In \eqref{op:bp_tv}, $\left \| \nabla_v x \right \|_1$ is the total variation of
the signal in the vertical direction and $\epsilon$ is a weighting parameter
which in this work is taken to be unity. The description of BPVV in
\eqref{op:bp_tv} can be viewed as the combination of CS and inpainting
(described below), in which the first term seeks a sparse solution while the
second term tries to diffuse pixel information. To solve the optimization
problem in \eqref{op:bp_tv} we use the split Bregman method
\cite{goldstein2009split}. The resulting algorithm is described in
Algorithm~\ref{ag:BPTV} where $\mu, \lambda, \gamma$ are scalar penalty
parameters that influence the convergence of the algorithm (though not the
solution found) and
\begin{equation}\label{df:shrink}
  \text{shrink}(x,\gamma) = \frac{x}{|x|}\max\left(|x|-\gamma,0\right),
\end{equation}
where all operations are element-wise.

\noindent 
\begingroup \setlength{\tabcolsep}{1pt}
\begin{figure*}[htbp]
  \centering
  \begin{tabular}{cccccc}
    \textit{\small original image} & \textit{\small BPDN} & \textit{\small BPVV} & \textit{\small detail of original} & \textit{\small detail of BPDN} & \textit{\small detail of BPVV} \\
    \includegraphics[width=0.16\textwidth]{figures-SBA/circulargrating_gt_framed}
                                   & \includegraphics[width=0.16\textwidth]{figures-SBA/circulargrating_40mu}		
                                                          & \includegraphics[width=0.16\textwidth]{figures-SBA/circulargrating_bptv_40mu}
                                                                                 & \includegraphics[width=0.16\textwidth]{figures-SBA/circulargrating_gt_zoomin}
                                                                                                                      & \includegraphics[width=0.16\textwidth]{figures-SBA/circulargrating_mu40_zoomin}
                                                                                                                                                       & \includegraphics[width=0.16\textwidth]{figures-SBA/circulargrating_bptv_40mu_zoomin}\\	
    \includegraphics[width=0.16\textwidth]{figures-SBA/anothergrating_gt_framed}
                                   & \includegraphics[width=0.16\textwidth]{figures-SBA/anothergrating_40mu}		
                                                          & \includegraphics[width=0.16\textwidth]{figures-SBA/anothergrating_bptv_40mu}	
                                                                                 & \includegraphics[width=0.16\textwidth]{figures-SBA/anothergrating_gt_zoomin}	
                                                                                                                      & \includegraphics[width=0.16\textwidth]{figures-SBA/anothergrating_40mu_zoomin}
                                                                                                                                                       & \includegraphics[width=0.16\textwidth]{figures-SBA/anothergrating_bptv_40mu_zoomin}	\\
    \includegraphics[width=0.16\textwidth]{figures-SBA/dnaforbptv_gt_framed}
                                   & \includegraphics[width=0.16\textwidth]{figures-SBA/dnaforbptv_40mu}		
                                                          & \includegraphics[width=0.16\textwidth]{figures-SBA/dnaforbptv_bptv_40mu}	
                                                                                 & \includegraphics[width=0.16\textwidth]{figures-SBA/dnaforbptv_gt_zoomin}	
                                                                                                                      & \includegraphics[width=0.16\textwidth]{figures-SBA/dnaforbptv_40mu_zoomin}
                                                                                                                                                       & \includegraphics[width=0.16\textwidth]{figures-SBA/dnaforbptv_bptv_40mu_zoomin}		
  \end{tabular}
  \caption{Reconstruction comparison between BPDN and BPVV. (first column)
    Original raster-scanned image. (second column) BPDN reconstruction from
    random, 40 pixel long $\mu$-paths with 25\% sampling. (third column) BPVV
    reconstruction from the same sub-sampled data. (remaining columns)
    Corresponding details from the red boxes indicated in the raster image. The
    results show that BPVV reduces the artifacts arising from the horizontal
    scans of the $\mu$-path pattern that appear in BPDN reconstructions.}
  \label{fig:BPTV_demonstration}
\end{figure*}
\endgroup
		
A comparison between BPDN and BPVV is shown in
Fig.~\ref{fig:BPTV_demonstration}. Images of three different samples (a circular
grating, a square grating and a DNA image) were acquired using an Agilent 5500
AFM with a standard raster scan (first and fourth columns in the figure). These
images were then resampled using a random $\mu$-path pattern of size 40 pixels
with 25\% of the pixels sampled. Images were reconstructed using BPDN (second
and fifth columns) and BPVV (third and sixth columns). The reconstruction
results show the artifacts apparent in the BPDN reconstruction are largely
mitigated using BPVV.
	

% =========================================================================
\section{Experimental setup} \label{sec:experimentalSetup}
% =========================================================================
Our experimental setup, illustrated in Fig.~\ref{fig:exp_setup_anderssonLab},
consists of an Agilent 5500 AFM, a cRIO-9082 embedded controller (National
Instruments), and a standard desktop computer. Through a breakout box, the
Agilent 5500 provides access to the $x$, $y$, and $z$-axis measurements as well
as the deflection signal (denoted in the following as $l$). When the Agilent
default software (PicoView) is set to open-loop mode, a $\pm 10$ V input on the
control box allows control of the $x$, $y$, and $z$-axis piezos. The N9524A
piezoelectric scanner in the Agilent 5500 provides a maximum scan area of
$90~\mu$m $\times$ 90~$\mu$m in the $xy$ (horizontal) direction and a range of
$20~\mu$m in the $z$ (vertical) direction. Initial engagement of the tip to the
sample is performed using the PicoView software before control is handed over to
the custom controllers.

TKTKT add stuff about nPoint stage and C300.
	
All control logic is programmed using LabVIEW 2017 and compiled to a Xilinx
Spartan-6 LX150 Field Programmable Gate Array (FPGA) inside the cRIO-9082. The
cRIO includes a 16-Bit, 100 kHz NI-9215 analog-to-digital input module and a
16-Bit, 100 kHz NI-9263 digital-to-analog output module. All control loops are
implemented using a 25 kHz sampling rate.

% =========================================================================
\section{Control}\label{sec:control}
The control structure for each axis is different. For the $X$-axis, we use a
state space controller. It is described fully in \cite{braker_afmmpc_2019} and
achieves a bandwidth of about 110 Hz, which is approximately 1/3 of the $X$-axis
resonance at 350 Hz.

The control law used for the $Y$-axis is a simple integral controllers of the
form
\begin{equation}
  \frac{U(z)}{E_Y(z)}=D(z) = \frac{K_Iz}{z-1},
  \label{eqn:dzI}
\end{equation}
where, the error signal $E_Y(z) = Y(z)-R_Y$.

\todo{Should I implement something better for $Y$-axis, e.g., cancel complex
  pole-zeros at 350 hz and add notch filter???}

The $Z$-axis control is slightly more complicated than the simple $Y$-axis
integral controller and is described in the following subsection.
\subsection{Z-axis Control Design}\label{sec:zaxis_cs_control}
% In Section~\ref{sec:zaxis-dist}, we eliminated or reduced many of the
% disturbances impinging on the $Z$-axis. This puts us in a position to
% increase the bandwidth without introducing building dynamics into our
% images.
The main challenge to increasing the $Z$-axis bandwidth is the bending mode
resonance at 215 Hz, which can be seen in the FRF from $u_Z$ to $Z_{d}$ in
Fig.~\ref{fig:z_control} (blue curve). We address this by inverting the complex
pole-zero pair at 215 Hz. There are multiple advantages to this method: first,
an inverse compensator still makes sense as a feedforward, open-loop compensator
if the feedback path is broken. This, situation can and does occur during parts
of the CS cycle. Second, as we discuss further below, the inverse compensator
requires no tuning as it is defined only by a fit to the measured FRF and this
can be almost completely automated.

\begin{figure}[t!]
  \centering
  \includesvg[width=1\columnwidth]{plots-afm-cs-final/figures/z_control_design.svg}
  \caption{}
  \label{fig:z_control}
\end{figure}
The entire $Z$-axis controller consists of a PI controller, $D_I(z)$ cascaded
with the resonance inversion, $D^{-1}(z)$. The inclusion of $D^{-1}$ allows us
to substantially increase the PI gain and achieve a closed-loop bandwidth of
about 480 Hz. This is the black curve in Fig.~\ref{fig:z_control}. For
comparison, the red curve in Fig.~\ref{fig:z_control} shows the closed-loop FRF
without the inverse compensator, which experiments indicate is unstable.

One challenge to this approach is the system gain and resonance at 215~Hz
changes not only day-to-day but also (and especially) when changing cantilevers.
This is illustrated in Fig.~\ref{fig:z_evolution}, which shows the FRF near the
mode for two three different cantilevers. Two are of the same make and model.

One cause for such different gains is variation in the length of the cantilever
(and possibly how the laser was adjusted). This is illustrated in
Fig.~\ref{fig:z_axis_gain}. Here, we consider a cantilever with length $h$ that
rests on the surface of some specimen with a nominal angle of $\theta$. In this
drawing, the laser is perpendicular to the sample surface. Given a vertical
perturbation $\Delta z$, we have that
\begin{equation}
  \Delta \alpha \approx \frac{d}{h} \Delta z,
\end{equation}
where $d$ is the distance from the cantilever to the detector and
$\Delta \alpha$ is the change in position of the incident laser spot on the
detector. Thus, given the same change in the vertical position, we will see a
larger change in the deflection signal for a shorter cantilever. In other words,
a short cantilever will exhibit a larger DC-gain in the $G_{u_Z,Z_d}$ transfer
function.
\begin{figure}
  \begin{minipage}[t]{.46\textwidth}
    % L B R T
    \includesvg[width=1\textwidth]{plots-afm-cs-final/figures/z_cant_evolution.svg}
    \caption{FRFs of three different cantilevers, showing substantial
      differences in the DC-gain and smaller, but still significant differences
      the frequency of the modes.}
    \label{fig:z_evolution}
  \end{minipage}
  \hfill
  \begin{minipage}[t]{.46\textwidth}
    % L B R T
    \includegraphics[width=1\textwidth]{plots-afm-cs-final/figures/z_axis_gain_schematic-crop.pdf}
    \caption{Schematic drawing of how the length of the cantilever affects the
      $z$-axis gain.}
    \label{fig:z_axis_gain}
  \end{minipage}
\end{figure}

Fig.~\ref{fig:z_evolution} shows the frequency response of $G_{u_Z,Z_d}$ near
the bending mode for three different cantilevers. Cantilevers (A) and (B) are
both AppNano SICON cantilevers and have a nominal length of 450 microns.
Cantilever (C) is an NCLR and has a nominal length of 225 microns.
Interestingly, the SICON cantilevers show substantial variation. Unfortunately,
the manufacture does not provide a tolerance on the nominal length, so it is
difficult to tease out whether that variation is due to manufacturing
variability or other factors not explained by this analysis.

Notably, the frequency of the pole and zero do appear to shift and our inversion
scheme is not robust to this. To deal with this, I have built a (mostly)
automated system ID routine into the imaging software. This routine is a
specialized version of the techniques described previously in
Chapter~\ref{chap:modeling}. First, it obtains an FRF of the bending mode over a
frequency range spanning 180~Hz to 240~Hz (the same set of points as in
Fig.~\ref{fig:z_evolution}). The input and output sinusoids are demodulated in
quasi-realtime as they come off the FPGA. Once this is complete, a 2-pole,
2-zero transfer function model of the form
\begin{equation*}
  \hat{D}(z) = k \frac{z^2 + b_1z + b_o}{z^2 + a_1z + a_o}
\end{equation*}
is fit to the FRF, using the same optimization scheme as \eqref{eqn:logfit}.
This is done inside the LabView VI by calling a custom DLL, which leverages
CMPFIT~\cite{cmpfit,markwardt_mpfit_2009}, a
Levenberg-Marquardt~\cite{more_levenberg_1978} routine written in C. To promote
a stable fit, the optimization enforces the necessary conditions that
\begin{align*}
  |a_1| &< 2, \quad  |a_o| < 1\\
  |b_1| &< 2, \quad  |b_o| < 1.
\end{align*}
The routine then checks that both the poles and zeros are indeed stabile, saves
the data to a JSON file and passes the transfer function parameters into the
imaging portion of the application which finally uploads the parameters into the
FPGA imaging bitfile. Optionally, one can by-pass this step and simply load the
previously saved JSON file.

One of the benefits of this approach is that the DC-gain of the transfer
function $G_{u_Z,Z_d}$ becomes (approximately) normalized to unity, which makes
re-tuning the control system after changing tips far easier to non-existent. The
benefit to automating the process is that it drastically reduces the friction
required to obtain an up-to-date compensator, which means that an up-to-date
compensator is more likely to be used~\cite{abramovitch_25years_2015}.


Fig.~\ref{fig:afm_bd_dinv} shows a block diagram if the $Z$-axis control loop.
We point out a small change from Chapter~\ref{chap:cs_init}: the addition of
$D^{-1}(z)$ changes the correct signal to use to represent the sample surface.
If we take the output of the entire compensator, $u_Z=D_ID^{-1}e_z$, then we
will get a lot of wiggles in the resulting signal. Rather, we need to take only
the output of the integrator to represent the surface.
% \begin{figure}
%   \centering
%   \includegraphics[width=1\columnwidth]{plots-afm-cs-final/figures/AFM_loop_Dinv-crop.pdf}
%   \caption{Block diagram of the $Z$-axis control loop with the inverse
%   compensator. }
%   \label{fig:afm_bd_dinv}
% \end{figure}
\begin{figure*}[t]
  \centering
  \includegraphics[width=0.75\textwidth]{plots-afm-cs-final/figures/AFM_loop_Dinv_antiwup-crop.pdf}
  \caption{Block diagram of the $Z$-axis control loop. The anti-windup is active
    for decision index $i\in\{0,1,5\}$, which correspond to states such that
    $r_Z=r_{Z,\textrm{up}}$.}
  \label{fig:afm_bd_dinv_final}
\end{figure*}


It should be noted that this control is sufficient to eliminate the limit-cycle
behavior described in Section~\ref{sec:aproachspeed}:

% \begin{figure}[tbhp!]
%   \centering\includegraphics[width=3.0in]{figures-SBA/uz_multiplexer_KI_anderssonLab2}
%   \caption{The $z$-piezo movement is driven by switching between feedback from
%   the $z$-axis position signal and from the deflection signal. The shaded area
%   represents the digital control system, while the non-shaded region is the
%   physical system.}
%   \label{fig:zMultiplex_anderssonLab}
% \end{figure}
	
As illustrated in Fig.~\ref{fig:zMultiplex_anderssonLab}, the voltage applied to
the $z$-piezo is adjusted by switching the feedback signal between the $z$-axis
position sensor (when the tip is off the surface) and the deflection signal
(when the tip is engaged). The difference equation associated with
\eqref{eqn:dzI} for the $z$-axis is
\begin{equation}
  u_z(k) = K_Ie(k) + u_z(k-1).
  \label{eqn:intdiff}
\end{equation}

In the $z$-axis position control mode, the $z$-axis position is controlled to a
fixed value $z_{ref}$ which, in general, is slightly above the sample surface.
The difference between the measured $z$-axis position and $z_{ref}$,
${e(k) = z_{ref} - z(k)}$, is used as the error signal. In the deflection
control mode, when the tip is in contact with the sample, the deflection signal
is controlled to a fixed value $l_{ref}$ to ensure a constant tip-sample force.
The sample surface acts as a disturbance to the control loop. Therefore, the
$z$-axis position measurements can be taken to represent the surface topography
of the sample. In this case, the difference between the measured deflection and
$l_{ref}$ is the error, $e(k) = l_{ref} - l(k)$.
	
We note that the $z$-axis control described here operates the AFM in constant
force mode during imaging. It is straightforward to perform scanning under
different AFM modes, such as tapping mode, simply by adjusting the signals and
control involved in the vertical loop.

% =========================================================================
\section{Implementation}\label{sec:implementation}
% =========================================================================
	
% =========================================================================
\subsection{State machine}
Implementing the $\mu$-path scheme involves operating the AFM in several
distinct stages. In the $xy$-direction, the system must transition from tracking
a step command (in the transition to a new measurement location) to tracking a
scan pattern. For our simple integral controller, this only involves changing
the reference signals $x_{ref}$ and $y_{ref}$. In the $z$-direction, the system
must transition between tip descent, surface scanning, tip retraction, and
maintaining a $z$-axis position. In addition, the $z$-piezo has to switch
between position control and deflection control with two fixed reference values,
$z_{ref}$ for position control and $l_{ref}$ for deflection control.
	
Transition between, and operation in, these different stages is implemented as a
simple state machine. The $x$, $y$, and $z$-axis control loops are operating in
parallel and independently. The state machine consists of the following states.

\begin{enumerate}
\item \textbf{Initialization:} In this state, the user uses the PicoView
  software to bring the tip to the sample with the stepper motor. Once the tip
  is engaged in the surface, the user switches the control from PicoView to the
  custom software in LabView. The system writes parameters including the
  locations of $\mu$-paths from a desktop file to a host-to-FPGA FIFO (first in,
  first out) buffer. Next, the closed-loop control of all the axes through the
  FPGA is enabled and the system performs the following operations.
  \begin{itemize}
  \item \textbf{Read} the starting location of the first $\mu$-path from the
    host-to-FPGA FIFO.
  \item \textbf{Set} the $z$-axis loop to position control mode.
  \item \textbf{Wait until} $|e_z(k)|$ reaches a settling criterion, then go to
    state 2.
  \end{itemize}
\item \textbf{$xy$-axis move:} The system moves the tip to the next target
  location.
  \begin{itemize}
  \item \textbf{Set} $x_{ref}$ and $y_{ref}$ to the beginning of the $\mu$-path.
  \item \textbf{Wait until} $|e_x(k)|$ and $|e_y(k)|$ reach a settling
    criterion, then go to state 3.
  \end{itemize}
\item \textbf{Tip engage:} The system drives the tip towards the sample surface.
  Here, we use a smaller $K_I$ for the $z$-piezo, which results in a slow
  descent and less windup while the tip is out of contact.
  \begin{itemize}
  \item \textbf{Set} the $z$-axis loop to deflection control mode.
  \item \textbf{Wait until} $|e_z(k)|$ reaches a settling criterion, then
    transition to state 4. \todo{Update this}
  \end{itemize}
\item \textbf{Scan:} In this state, the system moves the tip following the
  trajectory of the current $\mu$-path.
  \begin{itemize}
  \item \textbf{Update} $x_{ref}$, and $y_{ref}$ iteratively to follow the
    trajectory of the current $\mu$-path to the end. At the same time, write the
    $x,y,z,$ and deflection measurements to an FPGA-to-host FIFO.
  \item \textbf{Wait until} $|e_x(k)|$ and $|e_y(k)|$ reach the end of the
    $\mu$-path,
    % a settling criterion,
    then go to state 5.
  \end{itemize}
  % \item \textbf{Transition decision:} The tip is at the end of the $\mu$-path
  %   at this time. The location of the next $\mu$-path is read from the FIFO.
  %   In
  %   order to accelerate the scanning process, the decision to lift the tip or
  %   not is made in this state. The time $t_{xy\_scan}$ it takes to scan to the
  %   beginning of the next $\mu$-path is compared to
  %   $t_{\textit{z}up}+t_{xy}+t_{\textit{z}down}$. If the former is larger,
  %   then
  %   go to state 6; otherwise, go to state 2.
\item \textbf{Tip withdraw:} In this state, the system withdraws the tip in the
  $z$-direction. A large $K_I$ is used for the $z$-piezo here in order to save
  time.
  \begin{itemize}
  \item \textbf{Set} the $z$-axis loop to position control mode.
  \item \textbf{Set} the $z$ setpoint at a specified distance off the surface.
  \item \textbf{Wait until} $|e_z(k)|$ reaches a settling criterion, then go to
    state 2.
  \end{itemize}
\end{enumerate}
	
Two cycles of this sequential process are illustrated by the time series in
Fig.~\ref{fig:mupathsignaltrajectory}. The following subsections describe in
more detail several features of the CS scanning process.
	
% \begin{figure}[htbp!]
%   \centering
%     %   \includegraphics[width=3.0in]{figures-SBA/mupathsignaltrajectory}
%   \includegraphics[width=3.0in]{figures-SBA/mupathsignaltrajectory-swapped}
%   \caption{Two cycles of the $\mu$-path scanning process. The top two plots
%   are the lateral motion of the stage, the third is the $z$-sensor signal, and
%   the bottom is the deflection signal. Each stage of the state machine is
%   indicated by color.}
%   \label{fig:mupathsignaltrajectory}
% \end{figure}

% The first section discusses several modifications (i.e., differences to
% Chapter~\ref{chap:cs_init}) to the CS-scanning cycle. Section
% \ref{sec:post_process} discusses the post-processing techniques used. Section
% \ref{sec:rdi} develops a new metric to compare the relative damage done by
% different scans and Section~\ref{sec:psnr_ssim_limits} takes a closer look at
% the SSIM and PSNR metrics in an effort to understand what ''reasonable''
% numbers look like for AFM images taken with our AFM. Similarly,
% Section~\ref{sec:cs_sim} looks at some CS simulations (based on real raster
% scans) to help us understand what, ideally, we should see from the experiments
% performed in Section~\ref{sec:results:final}.

% Several cycles of the CS-scheme are shown in
% Fig.~\ref{fig:mupathsignaltrajectory}. Each of these differences is discussed
% more fully in the following three subsections.
\begin{figure*}[ht!]
  \centering \includesvg[scale=0.85]{plots-afm-cs-final/figures/cs_cycle.svg}
  \caption{Several CS cycles. Each state is indicated by color.}
  \label{fig:mupathsignaltrajectory}
\end{figure*}

% \begin{figure*}
%   \centering
%   \includesvg[width=.9\textwidth]{plots-afm-cs-final/figures/state-machine-final.svg}
%   \caption{Diagram of the state-machine implemented in the final design.}
%   \label{fig:sm_final}
% \end{figure*}

\subsection{Cantilever does not need to fully disengage.}
In the initial implementation, we insisted that the during tip-retraction
(state-5), the cantilever tip should fully break contact with the surface. Here,
we impose no such requirement. Rather, we change the $Z$-setpoint during
tip-retraction to only pull away far enough that the deflection signal stays low
(i.e., below the scanning setpoint), even while we run across the surface.

\textbf{Motivation} The motivation for this is two-fold. First, there is the
obvious advantage that, by not pulling the tip so away, there is less distance
to recover during the re-engagement. The original idea was to pull away from the
surface far enough to disengage, then step back down towards the surface by some
fraction of the step up. This proved impractical for several reasons. First, it
requires the control system to be able to detect when disengagement has
occurred. While preliminary experiments indicate it may be possible to achieve
this by detecting a change in the derivative of the deflection signal, the
actual value of the control signal which achieved dis-engagement proved to be
highly variable (ranging from about 0.6 v to over 3 volts (210 nanometers to
over a micron)) and sometimes saturating the control signal without
disengagement. Experiments indicate that the dis-engagement point depends on (at
least) the tip condition and the cantilever spring constant \footnote{The
  numbers cited used a cantilever with $k\approx .1$~N/m. Experiments using a
  cantilever with $k\approx 50$ N/m yielded much more consistent and much
  smaller needed dis-engagement control signals. However, cantilevers with such
  large spring constants are tapping-mode cantilevers, a mode which our AFM is
  incapable of. Scanning in contact mode with such a cantilever implies that a
  very large force is being applied to the sample.}.

Such large pull-away distances are problematic, especially for our AFM, which
lacks a $Z$-position sensor. The difficulty is that when the tip is re-engaged
with the sample from far away, this incites a large drift transient. This is
shown in Fig.~\ref{fig:z_drift} under the panels labeled ``big lift''. By
starting the re-engagement closer to the sample, this transient is much smaller.
Without a $Z$-position sensor, we have to let this drift die out or it will
corrupt the image. Thus, even if the step-up-down scheme could be made to work,
the overall time would still be limited by this drift. I made some effort to fit
an LTI model to the drift mode, so this affect could be removed in
post-processing. Unfortunately, as we saw in
Section~\ref{sec:modeling_remaining}, an LTI drift model does not adequately
describe the drift transient, because the drift rate seems to depend on the
control history.

A third advantage is that, given the \emph{same} absolute starting height above
the specimen, re-engagement is faster when the tip has not snapped free. The
reason for this is that when the cantilever snaps away from the specimen
surface, the deflection signal is essentially saturated. This gives the control
loop an artificially small error signal, which leads to a very sluggish descent.


\textbf{Justification} Ultimately the concern is damage to the sample and tip,
which is caused by the force imparted by the cantilever probe
\cite{clayton_review_2009}. The deflection signal, though un-calibrated (meaning
it does not directly lead to a value in $kN$), is proportional to that force,
because it is proportional to how much the cantilever is bent. I.e., a large
(towards $+\infty$) deflection implies the spring force of the cantilever is
pushing hard into the sample, thereby inducing damage. In a typical imaging
scenario, we choose a deflection setpoint, (say $0.15$~[v], as in
Fig.~\ref{fig:z_drift}), and have in some way decided that this is sufficient,
all else being equal, to not damage things too much.

Thus, rather than completely dis-engaging the tip from the surface, I propose to
simply move the setpoint towards $-\infty$ during the $XY$-move. If this new
setpoint is sufficiently far away, the deflection signal will stay below the
scanning setpoint, which we had decided was a ''safe'' value. The change in
setpoints is achieved via the multiplexer in Fig.~\ref{fig:afm_bd_dinv_final}
which selects either the scanning setpoint $r_{Z,s}$ or the retraction setpoint
$r_{Z,\textrm{up}}$ based the current state.

\begin{figure}[ht!]
  \begin{subfigure}{.48\textwidth}
    \includesvg[width=1\textwidth]{plots-afm-cs-final/figures/justify_small_lift.svg}
    \caption{ }
    % \label{fig: }
  \end{subfigure}
  \begin{subfigure}{.48\textwidth}
    \includesvg[width=01\textwidth]{plots-afm-cs-final/figures/justify_small_lift_zoom.svg}
    \caption{ }
    % \label{fig: }
  \end{subfigure}
  \caption{One CS-cycle with different dis-engagement heights. (a) Full view.
    (b) Zoomed-in view. Note the drift transient in the left panel of (b). }
  \label{fig:z_drift}
\end{figure}

This new scheme does not imply that the cantilever \emph{never} fully
disengages. Occasionally, the new setpoint is far enough away that the probe
does still snap away from the sample surface. When this happens, the integrator,
$D_I(z)$, will windup. This happens because typically, the desired setpoint
$r_{Z,\textrm{up}}$ is more negative than the free (disengaged) value of the
deflection. Thus, the sensor signal is essentially saturated and no matter how
far the $Z$-actuator tries to move, no change will result in the deflection,
leading to windup and a control signal that exceeds the actuator bounds. To deal
with this, we implement an anti-windup scheme for all states where
$r_Z=r_{Z,\textrm{up}}$, i.e., states (0), (1) and (5). This is illustrated in
the block diagram in Fig.~\ref{sec:modfied_cs_cycle}. During these states, the
output of the accumulator is saturated before its own internal feedback.

\subsection{The pre-scan}
\begin{figure*}
  \begin{subfigure}{.33\textwidth}
    \includesvg[width=1\textwidth]{plots-afm-cs-final/figures/zbounce_uz_example.svg}
    \caption{ }
    \label{fig: }
  \end{subfigure}
  \begin{subfigure}{.33\textwidth}
    \includesvg[width=1\textwidth]{plots-afm-cs-final/figures/prescan_uz_example.svg}
    \caption{ }
    \label{fig:uz_prescan}
  \end{subfigure}
  \begin{subfigure}{.33\textwidth}
    \includesvg[width=1\textwidth]{plots-afm-cs-final/figures/prescan_zbounce_PSD.svg}
    \caption{ }
    \label{fig: }
  \end{subfigure}
  \caption{(a) A ``$Z$-bounce'' experiment where the $XY$-stage is motionless
    and the cantilever is repeatedly engaged and disengaged. (b) One cycle of a
    tightly packed set of CS-cycles where the movement between each measurement
    location is 0. (c) PSDs from the portions of the trajectories which are
    colored blue and turquoise, averaged over many such cycles, demonstrating
    that the scan process itself reduces vibration in the $Z$-axis.}
  \label{fig:prescan_difference}
\end{figure*}
After the tip-descent, we begin scanning in a ``pre-scan'' phase (state (3)). To
motivate, this consider Fig.~\ref{fig:prescan_difference}. The left panel
represents what we call a ``$Z$''-bounce experiment, where the $XY$-stage stays
motionless, but we repeatedly retract and re-engage the tip with specimen,
exactly as we would in a CS scan. The right panel represents a single cycle from
specialized CS pattern, where the $\mu$-paths are still 500 nm long, but the
paths are separated by 0 pixels. In this case, the specimen is freshly cleaved
mica. The goal here is to eliminate (i) any oscillations which arise from the
$XY$-move (which, as we will show in Section ~\ref{sec:need_mimo}, can be
significant) and (ii) to have a perfectly flat surface to scan over so that
there are as few disturbances as possible entering the control loop. Finally,
the middle panel of Fig.~\ref{fig:prescan_difference} shows the PSDs of the left
and right panels (but averaged over many similar cycles). Note that the PSD is
only computed from the combination of the blue and turquoise signals.

What we see then, is that, for some unknown (to me) reason, scanning reduces the
amount of energy that shows up near 215 Hz. The idea behind the pre-scan is to
take advantage of this. The pre-scan phase is relatively long (150 samples),
especially for the faster scan rates. However, the pre-scan also allows us to
eliminate the over-scan at the end of the $\mu$-path, because by the time we are
ready to start taking data, the $X$-axis is already in quasi-steady-state. For
the current control design, the required over-scan is about 90 samples (which
can be computed from the steady-state error of the closed-loop system to a
ramp), which means the pre-scan incurs an effective overhead of about 60 samples
per $\mu$-path.

The tip settling (pre-scan) period could likely be much shorter if we a had a
$Z$-axis position sensor. Part of the reason it is so long is that, despite not
pulling the cantilever completely away from the sample, there is still a small
drift transient, as can be seen Fig.~\ref{fig:uz_prescan}. We would like to at
least give the major part of that drift transient time to die out. With a
position sensor, we would not care about the drift transient: the control signal
drifts to keep the $Z$-position constant.

\section{Post-processing}\label{sec:post_process}
Here, we discuss how we post-process the raster and CS images.

\subsection{Aligning images via cross-correlation}\label{sec:align}
To get metrics that are meaningful in any sense, we need the master image and
the comparee to be aligned. We take two steps to help ensure this. First, during
each batch of images, the nPoint stage remains under closed loop control.
Between images, the FPGA control loop idles in state-0, as discussed above in
Section~\ref{sec:idle_0}. This was not necessary duing the initial
implementation because then, the nPoint stage was controlled by the nPoint PI
controller. While this helps things substantially, there is still an alignment
issue between subsequent images, which I believe stems from the original Agilent
piezo tube, which is a three axis tube. The $XY$-axes of the tube are
uncontrolled during imaging, and in fact, after a firmware ``upgrade'', the
option to close the $XY$-loop has been grayed out in the PicoView software.

To help deal with this, we compare sub-slices of the master image and the
comparee. The results in this chapter use a sub-image obtained by removing a
border of 25 pixels from each edge. Thus, for the purposes of comparison, the
original 512$\times$512 pixel image becomes a 462$\times$462 pixel image. Let
$X$ represent the original, master image and ${\tilde X=X_{25:-25, 25:-25}}$,
using Python notation. To find the correct subslice of the comparee, we use a
two dimensional cross-correlation, which is defined as
\begin{equation}
  Z_{k,\ell} = \sum_{m=0}^{M-1}\sum_{n=0}^{N-1} \tilde{X}_{m-k, n-\ell}Y_{m-k, n-\ell}.
\end{equation}
where $Y$ is the comparee.

The index of the maximum value of the matrix $Z$
\begin{equation}
  (i^*,j^*) = \textrm{arg~} \max_{i,j}Z_{i,j}
\end{equation}
corresponds to the lower left corner of the correct subslice. Thus, the
comparisons will computed against $\tilde{Y} = Y_{i^*:i^*+487, j^*:j^*+487}$.
For an example of this in action, see the Fig.~\ref{fig:baseline_errors} in the
next section.

Unfortunately, the mis-alignment is not simply constant offset, but rather
drifts over time. This can be seen, e.g., in Fig.~\ref{fig:rast_aligned_0p5}.
Thus, while the technique described here mitigates the issue, it does not fully
eliminate the problem.



% \subsection{Dynamic de-trending}
% Because our AFM is not equipped with a $Z$-axis sensor, it is important to
% remove as much drift as possible from the $u_Z$ control signal, otherwise,
% this slow dynamic will corrupt the final image. This can partly be achieved by
% doing an image after the AFM has been on for a long time, e.g., 20 minutes,
% and especially after taking several slow raster scans. We can also remove some
% of the drift by fitting a curve
% \begin{equation}
%   d_k = \alpha + \beta t_k + \gamma \log\left(\frac{t_k}{0.1}\right)
% \end{equation}
% where $t_k$ is the time at the $k$-th sampling period and $\alpha$, $\beta$,
% and $\gamma$ are the parameters to be fit. This is a combination of a
% time-domain drift model \cite{Jung_open_loop_2000} and a linear fit which
% accounts for sample tilt. This line is fit to the entire set of $u_Z$ control
% data and subtracted.


\subsection{CS-reconstruction}\label{sec:tv_denoise}
Some of the noise in an image reconstructed via BPDN can be reduced by solving a
Total Variation (TV) de-noising optimization problem
\begin{equation}
  \min_{U} ||\nabla_xU||_1 + ||\nabla_yU||_1 + \mu||F - U||_2 \label{eqn:breg1}
\end{equation}
where $F$ is the image produced by the BPDN optimization and $U$ is the
de-noised image. This idea was heavily inspired by Yufan Luo's approach
\cite{luo_mupath_2019}. The difference is that (i) I do this for both $\nabla_x$
and $\nabla_y$, where he only does it for $\nabla_y$ and (ii) that he does it in
a single optimization which combines \eqref{eqn:breg1} with BPDN while I do it
in two separate optimizations. The optimization is solved using the Split
Bregman technique described in \cite{goldstein_splitbregman_2009} and my
implementation is publicly available \cite{L1c}.

\section{A New Metric}\label{sec:rdi}
\begin{figure}[t!]
  \centering
  \includesvg[width=1\columnwidth]{plots-afm-cs-final/figures/damage_illustration.svg}
  \caption{The RDI metric defined in \eqref{eqn:RDI} penalizes everything in the
    red shaded region and does not penalize anything in the un-shaded region.}
  \label{fig:damage_illustrate}
\end{figure}
In Chapter~\ref{chap:cs_init} that we employed two metrics, the SSIM and PSNR,
to compare different raster and CS scans to a master image. We took the master
image as a raster scan taken at a slow rate. These metrics only assess the
similarity of two images. However, in AFM imaging, pure image quality is not the
only concern, particularly for delicate samples. What is needed is a figure of
merit for how much damage was done to specimen during the imaging process. While
this is not really a concern while imaging a hard calibration grating, it plays
a prominent role biological samples~\cite{ando_highspeed_2008}. In general,
damage occurs while scanning into an uphill region, which results in a positive
deflection signal. The actual damage done by a given $Z_d$ will depend on the
spring constant of the cantilever and the softness of the specimen. Nonetheless,
we can use the positive deviations of $Z_d$ as a relative measure of damage
between different scan speeds or scanning methods, for a given cantiliver and
specimen. This motivates a metric we term the relative damage index (RDI), which
we define as
\begin{align}
  \text{RDI} &= \frac{1}{T_s}\sum_{k\in I} \frac{1}{k} \left(Z_{d,k} - r_{Z,s}\right)^2 \label{eqn:RDI}\\
  I &= \{k: k\in[0,~N-1],~ Z_{d,k}-r_{Z,s} > 0 \} \nonumber
\end{align}
where $N$ is the total number of samples in a given scan, $Z_{d,k}$ is the
deflection signal at sample $k$ and $r_{Z,\textrm{s}}$ is the scanning setpoint.
The RDI is basically the power in positive deflection. By excluding negative
values of the deflection, we do not penalize the CS algorithm while it is
re-engaging with the specimen \emph{unless} it overshoots the scanning setpoint,
which we do want to penalize. This is illustrated in
Fig.~\ref{fig:damage_illustrate}, which shows several CS cycles with a poorly
tuned controller. Only deflection signals in the red-shaded region will be
penalized by the RDI.

% Define the master image as $X$ and a reconstruction as $Y$, with each having
% $p\times p=n$ pixels. Stack the columns of each into the vectors
% $x, y\in\mathbb{R}^{n^2}$. Let $L$ be the dynamic range of the master image
% $x$. Then the PSNR is given by
% \begin{equation*}
%   \text{PSNR}(x,y) = 10\log_{10}\frac{L^2}
%   {\sqrt{\frac{1}{p^2} \sum_{i=1}^{n}( x_{i} - y_{i})^2}}.
% \end{equation*}
% The goal of the SSIM is to compare two image's structure, luminescence, and
% contrast and is built up from the means ($\mu_x$ and $\mu_y$), standard
% deviations ($\sigma_x$ and $\sigma_y$), and covariance ($\sigma_{xy}$) of the
% image vectors $x$ and $y$. The variation of the SSIM used in this thesis is
% defined as
% \begin{equation*}
%   \text{SSIM}(x,y) = \frac{(2\mu_x\mu_y + C_1)(2\sigma_{xy}+C_2)}
%   {(\mu_x^2 + \mu_y^2 + C_1)(\sigma_x^2 + \sigma_y^2 + C_2)}
% \end{equation*}
% where the constants $C_1$ and $C_2$ are regularizing constants to prevent
% singularity if, e.g., $\mu_x=\mu_y=0$. We use the default values suggested in
% \cite{wang_image_2004} of $C_1=(0.01L)^2$ and ${C_2=(0.03L)^2}$, where again,
% $L$ is the dynamic range. Both metrics have been used before to compare
% \emph{simulations} of CS reconstruction in the context of AFM
% \cite{oxvig_structure_2017, Luo_nano_2015}. We believe, however, that the
% numbers presented here in Table~\ref{tab:metrics} should be interpreted with
% some caution as it remains somewhat of an open question of how to best compare
% \emph{experimental} images from AFM.
\section{Limitations of SSIM and PSNR}\label{sec:psnr_ssim_limits}
What kind of numbers are reasonable to expect from the SSIM and PSNR metrics? We
mentioned in Chapter~\ref{chap:cs_init} that comparison of experimental images
was difficult (particularly for our experimental setup) because, e.g., Agilent
piezo tube drifts which gives a varying offset between images. However, we did
not try to tease apart to what extent the poor SSIM and PSNR numbers we reported
were due to simply the difference in scanning speeds vs things like image
mis-alignment.

In this section, we explore this subject by comparing images obtained under the
most ideal experimental conditions. Specifically, we look at the variation that
exists when comparing multiple raster scans taken at the same rate. Second, we
look at CS simulations where we sub-sample a raster image and reconstruct it.
The goal is to provide context to the numbers in
Section~\ref{sec:results:final}.

Recall that two images which are identical will yield an infinite PSNR and an
SSIM of 1.

\begin{figure}[t!]
  % ------------ 1.0 Hz ---------------------
  \begin{subfigure}{.48\textwidth}
    \includesvg[width=1\textwidth]{plots-afm-cs-final/figures/baseline_errors_noalign_1Hz.svg}
    \caption{1.0~Hz, no alignment.}
    \label{fig:rast_unaligned_1}
  \end{subfigure}
  \hfill
  \begin{subfigure}{.48\textwidth}
    \includesvg[width=1\textwidth]{plots-afm-cs-final/figures/baseline_errors_aligned_1Hz.svg}
    \caption{1.0~Hz, with alignment.}
    \label{fig:rast_aligned_1}
  \end{subfigure}
% 
  \caption{Errors between a ''master'' and 6 different raster scans, all taken
    sequentially and at a scan rate of 0.5 Hz. (a) Errors without alignment (b)
    The same data as above in (a) but aligned via cross correlation.}
  \label{fig:baseline_errors}
\end{figure}

In the first experiment, (Figs \ref{fig:rast_unaligned_0p5} and
\ref{fig:rast_aligned_0p5}) we took 6 raster scans of the same 5 by 5 micron
square of the CS-20NG sample grating. These scans were taken sequentially and
the NPXY100A remained in closed-loop throughout the process. Taking the first
scan as the master, we compute (without any alignment) the PSNR and SSIM
metrics. Fig.~\ref{fig:rast_unaligned_0p5} shows the error between the master
image and the remaining 6 images. Note that the best PSNR here is 12.13 and the
worst is 7.87. The best SSIM is 0.40 and the worst is 0.14.

Next, using the cross-correlation alignment technique discussed in
Section~\ref{sec:align}, we aligned the master image with a sub-slice of each of
the remaining 6 (the sub-slice removes 25 pixels from each side). Again, we
computed the PSNR and SSIM. These results are shown in
Fig.~\ref{fig:rast_aligned_0p5}, which again shows the errors between the
aligned master and the remaining 6 images. Although the situation is improved
compared to the un-aligned images, there is still noticable stretching,
particularly in the $Y$-direction. Notably, the with the alignment technique,
PSNR now ranges from SSIM 21.68 to 15.56 and the SSIM ranges from 0.67 to 0.50.


We repeat this experiment but this time use a 1~Hz scan.
Fig.~\ref{fig:rast_unaligned_1} shows the unaligned error images and
Fig.\ref{fig:rast_aligned_1} shows the aligned error images. Temporally, the
images are order left-to-right and top-to-bottom. Thus we can see, both visually
and via the metrics, that in Fig.~\ref{fig:rast_unaligned_1} the mis-alignment
grows over time (In the 0.5~Hz scans, it appears that after the third from, the
misalignment is so bad, the situation can't get any worse). In the aligned 1~Hz
images, our metrics are not only higher but also more consistent: and the SSIM
lies between 0.7 and 0.73 and the PSNR ranges from 22.51 to 24.99.

Due to this unfortunate drift, in Section~\ref{sec:results:final}, we will use a
1~Hz scan as the baseline image. Hopefully, this will give a decent trade-off
between having an image with minimal drift and scanning slowly enough that it
accurately represents the sample grating. Due to the improved $Z$-axis
bandwidth, a 1~Hz scan should still have better (or at least comparable) quality
than the 0.25~Hz scan taken in Chapter~\ref{chap:cs_init}. Additionally,
Fig.\ref{fig:rast_aligned_1} provides us with an (approximate) upper bound for
our expectations when looking at the experimental results in
Section~\ref{sec:results:final}.



\section{Experimental Results}\label{sec:results:final}
We took scans of the CS-20NG sample grating over an area with holes on a 500 nm
pitch. The holes are 20 nm deep. This is the same area imaged in
Chapter~\ref{chap:cs_init}. We scanned 5 micron by 5 micron images with 512
lines each, yielding 512 by 512 pixel images. As before, the $\mu$-path scans
are 500 nm long, as this allows us to work around the lack of a $Z$-axis sensor.
This yields a $\mu$-path size of 52 pixels.

The free value of the deflections signal was approximately -0.6 volts and we set
the scanning reference to $r_{Z,s}=-0.3$~volts and the withdraw reference to
$r_{Z,\textrm{up}}=-0.8$ volts. We set the $XY$-settle boundary to $\pm 0.05$
microns and required 20 samples within this boundary before moving to the
$Z$-engage state. The pre-scan length was 150 time steps. The control for the
$Z$-direction was as described in Chapter~\ref{chap:zaxis-improv}, and the
$X$-direction control used the SLF state-space (CZ) design of
Chapter~\ref{chap:mpc_slf} with one difference: the hysteresis model was
re-tuned for a $\pm$ 7 micron area. The $Y$-axis used a simple integral
controller, whose closed-loop FRF was shown in Fig.~\ref{fig:mimo_cl_frf}.

For the raster scans, we took scans at 1.0~Hz, 4.0~Hz, 5.0~Hz, 8.0~Hz and
10~Hz\footnote{For a 25~kHz sampling frequency, a 10~Hz raster scan at 512
  pixels is close to the upper limit because each pixel will only have
  approximately $ \frac{25000}{2\cdot 10 \cdot 512}\approx 2.4$ samples.}. The
CS scans we taken with scan velocities equivalent to raster scans of 1.0~Hz,
2.0~Hz, 4.0~Hz, 5.0~Hz and 8.0~Hz with sampling densities of 10~\% and 15~\%. As
in Chapter~\ref{chap:cs_init}, the cantilever was an AppNano SICON, with a
length of 450 microns and spring constant $k\in[0.02,~0.8]$~N/m.

The CS scans were reconstructed with BPDN (see \eqref{eqn:bpdn}) using the
log-barrier Newton method available in \texttt{L1c}\cite{L1c}. We set
$\epsilon = 0.1$ and the data remained scaled in terms of volts. We then
filtered the resulting image with the TV-denoising optimization described in
Section~\ref{sec:tv_denoise} with $\mu=100$.

For the raster scans, we discard data from the re-trace and divided the
remaining data into 512$\times$512 bins based on the \xc and \yc sensor
measurements. We then average the data in each bin to obtain the value of one
pixel. To remove the affects of piezo creep and sample tilt, we then de-trend
each individual line. De-trending each line introduces the artifacts we noted in
Section~\ref{sec:comp_raster}. To remove this effect, we select a column of
pixels on the left and right side of each image that does not cross any holes
(i.e., a vertical line through the flat area), register each scan line to a
common height along these columns. This step eases comparison with the CS scans
and was inspired by a feature in \texttt{SPIW} \cite{spiw}.

As a final step, we subtract the mean from each CS and raster image, so that
they can all be mapped to the same color intensity in the images. The color maps
represent a range of $\pm$20 nm.


The resulting images are shown in Fig.~\ref{fig:resultsF1_images}. The rows of
pixels indicated by the red lines are shown Fig.~\ref{fig:pixel_rows}. Due to
the improved $Z$-axis bandwidth and the image alignment issues described in
Section~\ref{sec:psnr_ssim_limits}, we use the 1.0~Hz raster scan as the master
image. Fig.~\ref{fig:resultsF1_errs} shows the error between aligned sub-slices
of the 1.0~Hz image and the remaining scans. The sub-slice removed a 30 pixel
boundary from each edge. Despite this alignment, the middle pixels tend to show
a smaller error than those towards the top and bottom, which is similar to the
affect observed before in Fig.~\ref{fig:baseline_errors}.

For each raster scan and each CS scan, we computed the SSIM, PSNR (both with
default parameters) using the 1~Hz scan as the master and also computed the RDI
metric. These are collected in table ~\ref{tab:rast_vs_cs_v1}.

\begin{figure*}
  \includesvg[scale=1]{plots-afm-cs-final/figures/cs_raster_images_4-26-2019}
  \caption{Raster and compressed sensing images of a 5 micron square area of the
    CS-20NG grating. All images are 512$\times$512 pixels. The rows of pixels
    indicated by the red line are shown in \ref{fig:pixel_rows}.}
  \label{fig:resultsF1_images}
\end{figure*}

% \begin{figure*}
%   \includesvg[scale=1]{plots-afm-cs-final/figures/cs_raster_images_err_4-26-2019}
%   \caption{Errors between aligned sub-slices of the 1.0~Hz raster image and
%   the remaining scans. The original scans are shown in
%   Fig.~\ref{fig:resultsF1_images}.}
%   \label{fig:resultsF1_errs}
% \end{figure*}


\begin{table}[t!]
  \centering
  \caption{Performance metrics for the scans taken on 4-26-2019. All CS images
    filtered with TV denoising, $\mu$=100.}
  \begin{tabular}{cccccc}
    type &  rate (Hz) & PSNR & SSIM & time [s] & damage\\
\toprule
raster & 1.00 & Inf & 1.00 & 512.0 & 1.06\\
raster & 4.00 & 21.96 & 0.72 & 128.0 & 8.11\\
raster & 5.00 & 22.07 & 0.72 & 102.4 & 10.86\\
raster & 8.00 & 22.52 & 0.71 & 64.0 & 18.58\\
raster & 10.00 & 21.23 & 0.65 & 51.2 & 23.70\\
CS (10.02~\%) & 1.00 & 20.50 & 0.69 & 34.8 & 1.21\\
CS (10.02~\%) & 2.00 & 20.89 & 0.70 & 22.0 & 2.77\\
CS (10.02~\%) & 4.00 & 20.43 & 0.68 & 15.6 & 6.57\\
CS (10.02~\%) & 5.00 & 20.90 & 0.69 & 14.3 & 8.21\\
CS (10.02~\%) & 8.00 & 20.58 & 0.67 & 12.5 & 13.13\\
CS (15.02~\%) & 1.00 & 21.24 & 0.73 & 52.1 & 1.15\\
CS (15.02~\%) & 2.00 & 21.50 & 0.73 & 32.9 & 2.67\\
CS (15.02~\%) & 4.00 & 21.42 & 0.72 & 23.3 & 6.62\\
CS (15.02~\%) & 5.00 & 21.22 & 0.72 & 21.4 & 8.11\\
CS (15.02~\%) & 8.00 & 20.92 & 0.70 & 18.6 & 12.93\\

  \end{tabular}
  \label{tab:rast_vs_cs_v1}
\end{table}

\begin{table}[t!]
  \centering
  \caption{Breakdown of state times for the CS scans listed in
    Table~\ref{tab:rast_vs_cs_v1}. All times are in seconds.}
  \label{tab:final_state_times}
  \begin{tabular}{ccccccc}
    description & move & engage & tip-settle & scan & tip-up & total\\
\toprule
1.0 Hz, 10.02~\% &4.33 & 1.07 & 3.06 & 25.85 & 0.48 & 34.80\\
2.0 Hz, 10.02~\% &4.30 & 1.07 & 3.06 & 13.08 & 0.48 & 21.99\\
4.0 Hz, 10.02~\% &4.31 & 1.08 & 3.06 & 6.70 & 0.48 & 15.63\\
5.0 Hz, 10.02~\% &4.28 & 1.07 & 3.06 & 5.41 & 0.48 & 14.31\\
8.0 Hz, 10.02~\% &4.34 & 1.09 & 3.06 & 3.50 & 0.48 & 12.46\\
1.0 Hz, 15.02~\% &6.45 & 1.59 & 4.58 & 38.76 & 0.72 & 52.11\\
2.0 Hz, 15.02~\% &6.38 & 1.60 & 4.58 & 19.60 & 0.73 & 32.90\\
4.0 Hz, 15.02~\% &6.35 & 1.61 & 4.58 & 10.04 & 0.73 & 23.32\\
5.0 Hz, 15.02~\% &6.33 & 1.62 & 4.58 & 8.11 & 0.72 & 21.37\\
8.0 Hz, 15.02~\% &6.41 & 1.64 & 4.58 & 5.25 & 0.72 & 18.60\\

  \end{tabular}
\end{table}

\begin{figure*}[t!]
  \includesvg[width=1\textwidth]{plots-afm-cs-final/figures/cs_raster_pixel_rows_3-20-2019.svg}
  \caption{Rows of pixels, as indicated by the red lines in
    Fig.~\ref{fig:resultsF1_images}. For clarity, not all images are included.}
  \label{fig:pixel_rows}
\end{figure*}

Unfortunately, the SSIM and PSNR metrics are not particularly illuminating. It
is easiest to see this by plotting PSNR and SSIM against total acquisition time,
which is shown in Fig.~\ref{fig:time_ssim_psnr}. There are several
inconsistencies. First, the 4~Hz raster scan has the lowest PSNR of all the
raster scans, while the 8~Hz scan has the highest. Second, all of the 15\% CS
scans except the 8~Hz version have as good or better SSIM numbers than the
raster scans. It makes no sense for a CS scan at a 5~Hz rate to have better
quality than a raster scan at a 5~Hz rate. It is interesting to also compare
these numbers to the batch of 1~Hz raster scans we compared in
Section~\ref{sec:psnr_ssim_limits}. The blue shaded bands of
Figs~\ref{fig:time_ssim_psnr} indicate the mean of the SSIM and PSNR metrics
from Section \ref{sec:psnr_ssim_limits} plus or minus one standard deviation.
This seems to be further evidence that, at least for our experimental setup,
these numbers are not especially meaningful.

\begin{figure*}[t!]
  \begin{subfigure}{1\textwidth}
    \includesvg[scale=1]{plots-afm-cs-final/figures/cs_rast_time_vs_ssim_psnr.svg}
    \caption{(left) SSIM as a function of total acquisition time. (right) PSNR
      as a function of total acquisition time.}
    \label{fig:time_ssim_psnr}
  \end{subfigure}
  
  \begin{subfigure}{1\textwidth}
    \includesvg[scale=1]{plots-afm-cs-final/figures/cs_rast_damage.svg}
    \caption{(left) RDI as a function of scan rate for raster scans and CS scans
      at 10\% and 15\% sampling. (right) RDI as a function of total imaging
      time.}
    \label{fig:time_damage}
  \end{subfigure}

  \label{fig:time_vs_metrics}
\end{figure*}

Where CS seems to really shine is if we look at the RDI for a given time or scan
rate. This is plotted in Fig.~\ref{fig:time_damage}. For example, the 10~Hz
raster scan, which takes about 51 seconds, gives an RDI of 23.7. All of the CS
scans, except for the 15\% scan at 1~Hz (which is only 0.9 seconds longer) are
faster than the 10~Hz raster scan, yet have much lower RDI values, ranging from
1.21 to 13.13. Unsurprisingly, the relationship between scan speed and RDI is
quite linear. For the raster scans, the left panel of Fig.~\ref{fig:time_damage}
just plots the inverse relationship of the right panel, since scan time is
inversely proportional to raster frequency. This is not quite the case for the
CS scans due to the sub-sampling and the fixed overhead which does not scale
with scan rate (e.g., the amount of time for the $XY$-move etc remains
essentially constant for a given scan density).

What is surprising is that for the same scan rate, the CS scans have a lower RDI
than a raster scan at the same rate. One possibility is that the probability
that a given $\mu$-path will start on the flat portion of the grating, scan
\textit{into} a hole and then the tip withdraws before exiting the hole is
higher than the probability that a scan starts in a hole and exits the hole.
This would result in a lower RDI, because scanning \textit{out} of hole is what
the RDI penalizes.

From the left panel of Fig.~\ref{fig:time_damage}, we see that for a fixed RDI,
CS has a faster imaging acquisition rate. The amount of improvement is best if
we need a very low RDI: for example, if we want to hold the RDI below 2, then we
have to take the 1~Hz raster scan (512 seconds) while we can achieve the same
RDI via CS in about 52 seconds with the 15\% sampling, which is a speed
improvement of nearly 10. The higher of an RDI we are willing to tolerate, the
advantage of CS over raster narrows. For example, the 4~Hz raster scan and 5~Hz,
15\% CS scan have comparable RDI, but now CS is only about 6 times faster. This
is a result of the constant overhead imposed by the engage/disengage and
$XY$-move states in CS.

\section{Conclusions}
The main conclusion of this chapter is that with CS, it seems that we can
acquire images faster yet (putatively) with less specimen damage than is
possible with raster scanning, which we showed by introducing a new metric, the
RDI. Of course, a natural criticism is that we did not validate the RDI against
delicate specimens. We leave this as an area for future work.

A more minor takeaway is that comparing experimental AFM images is difficult,
especially for systems like ours which cannot control the $XY$-axes of the piezo
tube.

One final point that should be mentioned is that, by using an $X$-axis
controller designed for tracking step inputs, we have unfairly disadvantaged
raster scanning. Due the faster harmonic decay of a triangle wave compared to a
series of step inputs, in principle a higher bandwidth $X$-axis controller could
be used for raster scanning because the cross-coupling concerns discussed in
Section~\ref{sec:cz_cr_cs} would play a smaller role. While this will be
discussed a bit further in the next chapter, for now we note that the most
prominent way the lower bandwidth $X$-axis controller has limited raster
scanning is by chopping off the left and right edges of the images, which can be
seen in Fig.~\ref{fig:resultsF1_images} and \ref{fig:pixel_rows}. However, the
metrics are computed on a subslice that is large enough to eliminate this
region. Additionally, the RDI is mostly affected by the $Z$-direction bandwidth.
    

%%%%%%%%% then the Bibliography, if any %%%%%%%%%
% \bibliographystyle{plain} % or "siam", or "alpha", etc.
\bibliographystyle{IEEEtran} % or "siam", or "alpha", etc.
% \nocite{*} % list all refs in database, cited or not
\bibliography{/home/arnold/bib_pdf/main_bibliography,Andersson_bibs,Yufans_bibs} % Bib database in "refs.bib"


%%%%%%%%% then the Appendices, if any %%%%%%%%%
\end{document}

